\section{More}
\begin{frame}
  \frametitle{More}
  Based on type theory instead of set theory:
  \[
    \begin{array}{c|c}
      \mathrm{set\ theory} & \mathrm{type\ theory}\\
      \hline
      & + : \mathrm{nat} \rightarrow \mathrm{nat} \rightarrow \mathrm{nat}\\
      \{ n \in \mathbb{N}\,|\, \phi(n) \in \mathbb{P}\} & 3 + 4 : \mathrm{nat}
    \end{array}
  \]

  \begin{itemize}
    \item we can mix proofs and programs
    \item reasoning on programming languages made simpler
  \end{itemize}
\end{frame}

% \begin{frame}
%   \frametitle{Théorie des types}
%   \begin{itemize}
%     \item $12 : \mathrm{nat}$
%     \item $+ : \mathrm{nat} \rightarrow \mathrm{nat} \rightarrow \mathrm{nat}$
%     \item $\mathrm{True} : Prop$
%     \item $\forall (A : \mathrm{Prop}),\, A \Rightarrow A : Prop$
%     \item $\mathrm{nat} : \mathrm{Set}$
%     \item $\mathrm{Set} : \mathrm{Type}$
%     \item $\mathrm{Prop} : \mathrm{Type}$
%     \item $\mathrm{Type} : \mathrm{Type}$
%   \end{itemize}
% \end{frame}

% \begin{frame}
%   \frametitle{Correspondance preuves -- programmes}
%   C'est le principal intérêt de la théorie des types. Analogies~:
%   \[
%     \begin{array}{|c|c|}
%       \hline
%       \mathrm{Set} & \mathrm{Prop}\\
%       A \times B & A \land B\\
%       A \cup B & A \lor B\\
%       A \rightarrow B & A \Rightarrow B\\
%       \forall (x : A),\, B(x) & \forall (x : A),\, B(x)\\
%       \mathrm{Programmes} & \mathrm{Preuves}\\
%       \mathrm{Types} & \mathrm{Théorèmes}\\
%       \hline
%     \end{array}
%   \]
% \end{frame}

% \begin{frame}
%   \frametitle{Correspondance preuves -- programmes}
%   \begin{itemize}
%     \item la définition de la logique découle de celle des programmes
%     \item il est possible de mélanger preuves et programmes:
%       \begin{itemize}
%         \item pour prouver des programmes
%         \item pour faire des programmes qui prouvent
%       \end{itemize}
%   \end{itemize}
% \end{frame}

% \begin{frame}
%   \frametitle{Logique intuitionniste}
%   \begin{itemize}
%     \item tout ce qui est prouvé doit pouvoir être contruit à partir des hypothèses
%     \item une preuve de $\exists (x : A),\, P(x)$ permet de calculer $x$
%     \item en particulier~: pas de tiers exclu, raisonnement par l'absurde, axiome du choix
%     \item tiers exclu et axiome du choix activable en option
%   \end{itemize}
% \end{frame}

% \begin{frame}
%   \frametitle{Preuves réflexives: idée}
%     Écrire un programme qui résout une classe de problèmes, en calculant si un problème est vrai ou faux, et le prouver correct.
% \end{frame}
% \begin{frame}
%   \frametitle{Preuves réflexives: exemple}
%   Pour un certain type d'équations différentielles, on écrit un programme en \textsc{Coq} qui renvoie \emph{vrai} si il existe une solution positive. On prouve que quand il renvoie \emph{vrai} il existe effectivement une solution positive.

%   Alors pour toute équation différentielle de ce type admettant une solution positive, on peut prouver que c'est le cas automatique. Il suffit d'exécuter le programme précédant et d'attendre le résultat.
% \end{frame}
