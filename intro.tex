\section{Introduction}
\begin{frame}
  \frametitle{Le logiciel Coq}
  Un language pour:
  \begin{itemize}
    \item énoncer des théorèmes
    \item écrire des preuves formelles
    \item écrire des algorithmes
  \end{itemize}

  Un environnement pour:
  \begin{itemize}
    \item raisonner de façon interactive
    \item organiser / distribuer les développements
  \end{itemize}
\end{frame}

\begin{frame}
  \frametitle{Développements importants}
  \begin{itemize}
    \item théorème des quatre couleurs (Gontier 04)
    \item théorème de Feit -- Thompson (Gontier \& all, 12)
    \item compilateur \textsc{C} CompCert (Xavier Leroy \& all)
    \item bibliothèque \emph{Bedrock} de vérification de programmes bas niveau
  \end{itemize}
\end{frame}

\begin{frame}
  \frametitle{Historique}
  \begin{itemize}
    \item Calcul des Constructions (\textsc{CoC}) par \emph{Thierry Coquand} (85)
    \item implémentation du \textsc{CoC} donnant lieu à \textsc{Coq}
    \item Calculus of Inductive Constructions (\textsc{CiC}) par \emph{Christine Paulin} (91)
  \end{itemize}
\end{frame}

\begin{frame}
  \frametitle{Logique}
  \begin{itemize}
    \item théorie des types plutôt que ensembles
    \item logique intuitionniste et constructive
    \item axiomes du tiers exclu, du choix optionels
  \end{itemize}
\end{frame}
